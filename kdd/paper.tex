\documentclass{article}

\usepackage{arxiv}

\usepackage[utf8]{inputenc} % allow utf-8 input
\usepackage[T1]{fontenc}    % use 8-bit T1 fonts
\usepackage{hyperref}       % hyperlinks
\usepackage{url}            % simple URL typesetting
\usepackage{booktabs}       % professional-quality tables
\usepackage{amsfonts}       % blackboard math symbols
\usepackage{nicefrac}       % compact symbols for 1/2, etc.
\usepackage{microtype}      % microtypography
\usepackage{graphicx}       % images
\usepackage{mathtools}
\usepackage{parselines}
\usepackage{biblatex}
\usepackage{verbatim}

\title{CHESS: Anomaly Detection}
\author{
    Thomas~J.~Howard~III \\
    Department of Computer Science and Statistics\\
    University of Rhode Island \\
    Kingston, RI 02881 \\
    \texttt{thoward27@uri.edu}
    \And
    Najib~Ishaq \\
    Department of Computer Science and Statistics\\
    University of Rhode Island\\
    Kingston, RI 02881 \\
    \texttt{najib\_ishaq@zoho.com}
}

\bibliography{references}

\begin{document}
\maketitle

\begin{abstract}
In this paper we use CHESS for introspective anomaly detection.
Using the underlying clustering mechanism from CHESS, along with some manually supplied stopping criteria, we present strong evidence that CHESS is able to accurately map the underlying manifold of the data and that we are able to use this manifold and its inherent properties to successfully flag anomalous data.
We empirically demonstrate the effectiveness of our method using several synthetic data sets.
Finally, we explore some future opportunities that could greatly increase the scope of this work.
\end{abstract}

\keywords{
Anomaly Detection
\and Outlier Detection
\and Novelty Detection
\and Manifold Learning
\and CHESS
}

\section{Introduction}
\label{sec:introduction}

Detecting anomalies or outliers from a distribution of data is a well-studied problem in machine learning. When data occupy easily-described distributions, such as Gaussian, the task is relatively easy, requiring only that one identify that a datum is sufficiently far from the mean or median. However, in ``big data'' scenarios, data occupy high-dimensional spaces that do not behave intuitively. Furthermore, data may obey the ``manifold hypothesis''~\cite{fefferman2016testing}, occupying a low-dimensional manifold in the higher-dimensional embedding space. Detecting anomalies or outliers in such a landscape is not so easy; in particular, correctly identifying an anomalous datum that sits between ``branches'' of a tree-like manifold presents a challenge.

Anomalies (data which don't belong to a particular distribution) and outliers (data which represent extrema of a distribution) can arise from many sources: errors in measurement or data collection; novel, previously-unseen instances of data; normal behavior evolving into abnormal; or adversarial attacks as seen in examples of malicious inputs to machine-learning systems~\cite{schoolbus-ostrich}



%% The below all goes AFTER the related works.

We introduce a novel technique, Clustered Learning of Approximate Manifolds (CLAM). This
approach uses a divisive hierarchical clustering approach to learn a manifold in a
Banach space~\cite{} defined by a distance metric. In actuality, our requirements are
less strict; the space can be defined by a distance \textit{function} that does not obey
the triangle inequality, though this is not always optimal. Given a learned manifold, we
can almost trivially implement several anomaly-detection algorithms; in this manuscript,
% NMD note: this cannot stay, much as I love it.
we present a collection of five such algorithms implemented on CLAM: CHAODA (Clustered Hiearchical Anomaly and Outlier Detection Algorithms).


The manifold learning component derives from prior work aimed at accelerating the task of approximate search on large data sets of high dimension, CHESS~\cite{ishaq2019entropy}. CLAM begins by divisively clustering the data until every point is within its own cluster, as a singleton.
CLAM then delineates \textit{layers} of clusters at each depth in the tree.
Each layer comprises all clusters that would have been leaf nodes if the tree building were to have been halted at the given depth.

CLAM then build a graph for each layer in the tree by creating edges between clusters that have overlapping volumes.
This process effectively learns the manifold on which the data lie at various resolutions, given by the depth of the layer. This is analogous to a ``filtration'' in computational topology~\cite{carlsson2009topology}.
Once we have learned a manifold, one can ask about the cardinality of various clusters at different depths, how connected a given cluster is, or even how often a cluster is visited over many different random walks across the manifold.

We test our methods on 26 real-world datasets. 
Each dataset contains a different number of anomalous data, for a different domain.
We consider several different definitions for outliers and anomalies: \textbf{distance-based}, examining several classical distance-based definitions of outliers, relying on CLAM's use of distance to cluster data; \textbf{density-based}, examining the cardinality of clusters, under the hypothesis that clusters with lower cardinality are more likely to contain outliers; \textbf{graph-based}, examining several graph-theoretic methods for anomaly detection, given graphs constructed from layers of clusters.

Historically, clustering approaches have suffered from one several problems.
The most common deficiencies are: the effective treatment of high dimensionality, interpretability of results, or scalability to exponentially-growing datasets ~\cite{rakesh_agrawal_automatic_1998}.
With CLAM, we have largely alleviated these problems while also decoupling search time from the size of the data being searched.
% unsure how much we can or want to say about search time given the page constraints
\section{Related Works}
\label{sec:realted_works}

In this section, we consider distance-based, clustering-based, and graph-based approaches.

\subsection{Clustering-Based Approaches}

Clustering data is universally centered around techniques for grouping points in a way that provides value.
This is done by assigning similar points to the same cluster.
Once these clusters are assigned, they can be used to check if there are any outliers present within the data.
The clusters can also be used to check if new points are anomalous.
For example, for any given point, one can check if it lies within, or near, any cluster; if not, the point can be marked as an anomaly.
To determine if the data contain any outliers, one can also examine the cardinality of various clusters and/or how dissimilar the clusters are to each other.
For example, if many clusters of large cardinality are present, along with relatively few clusters of very low cardinality, the clusters of low cardinality may contain outliers.

We summarize several major approaches to clustering data.
Note that there have been relatively few advancements in clustering techniques over the past decade~\cite{wang_progress_2019}. This may explain, or by attributable to, the observably poor performance, so far, of clustering data in high dimensional space~\cite{zhang_advancements_2013}.

\subsubsection{Distance-Based Clustering}

Distance-based clustering approaches rely on some distance measure to partition the data into some number of clusters.
Within this approach, the numbers of clusters and/or the sizes of clusters are predefined: either user-specified, or chosen at random~\cite{wang_progress_2019}.
Some examples of distance-based clustering are: 
K-Means~\cite{macqueen_methods_nodate}, 
PAM~\cite{kaufman_finding_nodate}, 
CLARANS~\cite{ng_efficient_nodate}, 
CLARA~\cite{kaufman_finding_nodate}, 
etc.

\subsubsection{Hierarchical Clustering}

Hierarchical clustering methods typically utilize a tree-like structure, where points are allocated into leaf nodes~\cite{wang_progress_2019}.
These tree-like structures can be created bottom-up (agglomerative clustering), or top-down (divisive clustering)~\cite{rakesh_agrawal_automatic_1998}.
The major drawback of these methods is the high cost of pairwise distance computations at each level of the tree.
Examples of hierarchical clustering include: 
MST~\cite{charles_zahn_graph_1971}, 
CURE~\cite{noauthor_cure:_nodate}, 
CHAMELEON~\cite{karypis_chameleon:_nodate}, 
etc.

\subsubsection{Density-Based Clustering}

Density-Based clustering methods rely on finding areas of high point-density separated by areas of low point-density.
These algorithms generally do not work well when normal data are sparse.
Some examples of density-based clustering algorithms are: 
DBSCAN~\cite{ester_density-based_nodate}, 
DENCLUE~\cite{Hinneburg1998Effic-5816}, 
etc.

\subsubsection{Grid-Based Clustering}

Grid-based clustering works via segmenting the entire space into a finite number of cells and then iterating over the cells to find regions of higher density.
Utilizing the grid structure for clustering means that these algorithms typically scale better to larger datasets.
Some examples of grid-based clustering: 
STING~\cite{sting:wang}, 
Wavecluster~\cite{Wavecluster:Sheikholeslami:2000}, 
DCluster,  % Paper does not exist????
CLIQUE~\cite{rakesh_agrawal_automatic_1998}, 
etc.

\subsection{Distanced-Based Approaches}

Distance-based methods attempt to find anomalous points via distance comparisons.
These methods largely employ k-Nearest Neighbors as their substrate~\cite{wang_progress_2019}.
They proceed in the following, slightly different, ways:

\begin{enumerate}
    \item Points with less than $p$ other points within a distance $d$ are outliers.
    \item The top $n$ points whose distances to their $k^{th}$-nearest neighbor are the greatest are outliers.
    \item The top $n$ points whose average distances to their $k^{th}$-nearest neighbor are the greatest are outliers.
\end{enumerate}


% CLIQUE Summary
\begin{comment}
CLIQUE scales linearly with input records, but does identify subspaces within the original dataset.
CLIQUE clusters based on subspaces it identifies via density estimates, it does not cluster in the full dimensionality of the input data.
This seems to be the premonition of manifold finding.
Restricted look to subspaces of the original set, not fictitious subspaces dreamed up via techniques such as Principle Component Analysis.
CLIQUE finds areas of higher density, which are the clusters.
Begins by partitioning the full space into units then estimates density within that cell.
All clusters are axis-parallel hyper-rectangles. Each cluster is a union of adjacent dense cells.
Compact representation is then found by covering the cluster with a minimal number of maximal, possibly overlapping rectangles and describing the cluster as the union of said rectangles.
CLIQUE is also tolerant to incomplete data points.
Requires density threshold and number of intervals over the full space as parameters.
Categorical data is ordered arbitrarily, with an empty interval placed in-between ever pair.
Is deterministic, always yields the same results.
Runs in $O(c^k + m k)$ where $c$ is some constant, $k$ is the dimensionality, and $m$ is the number of points.
Tested on dimensionality up to 50.
In experiments run for this paper, BIRCH failed starting at 20 dimensions, DBSCAN at 8, CLIQUE made it to 50 without failure.
\cite{rakesh_agrawal_automatic_1998}
\end{comment}

\section{Methods}
\label{sec:methods}

For this initial work, we have extended CHESS to a Manifold Learning algorithm.
We use this method to address Anomaly and Outlier Detection.
We use Distance-Based, Clustering-Based and Graph-Based procedures for detecting anomalous and outlier points/clusters.

% TODO: Transition these definitions to algorithms

\subsection{Distance-Based Outliers}

\begin{enumerate}
    \item Sort all points $p \in \mathbb{P}$, where $\mathbb{P}$ is the data, by $f \equiv |B_D(p, r)|$ in ascending order.
    \begin{itemize}
        \item If needed, increase $r$ to break ties.
        \item The points with the smallest values of $f$ are the outliers.
    \end{itemize}
    \item Sort all points $p \in \mathbb{P}$ in descending order by the distance to their $k^{th}$ nearest neighbor.
    \begin{itemize}
        \item Consider how this distance changes as $k$ in increased.
        \item The points with the highest such distances are the outliers.
    \end{itemize}
\end{enumerate}

\subsection{Hierarchical-Clustering-Based Outliers}

\begin{enumerate}
    \item For a $parent$ cluster and its $left$ and $right$ child-clusters (assume without loss of generality that $|left| \leq |right|)$ define $f \equiv \frac{|left|}{|parent|}$. If $f \ll 0.5$ then points in the $|left|$ cluster are outliers.
    \begin{itemize}
        \item Use this definition recursively down the tree. The number of levels in the tree at which a point is labelled an outlier gives us a measure of the ``anomalousness'' of that point.
    \end{itemize}
\end{enumerate}

\subsection{Graph-Based Outliers}

\begin{enumerate}
    \item The Outrank algorithm~\cite{moonesinghe_outrank:_2008} i.e. on a given graph-representation of data, initiate a random walk. The clusters that are visited least often are the outlier clusters.
    \item Define the ``k-neighborhood'' of a clusters $c$ as the clusters that can be reached from $c$ in $k$ steps. Investigate how $|k-neighborhood|$ increases as $k$ in increased.
    \begin{itemize}
        \item $|k-neighborhood|$ should increase by $k^d$ where $d$ is \textit{local fractal dimension} of the data at the length-scale of the radii of the clusters that form the graph.
        \item If the increase in $|k-neighborhood|$ of a cluster $c$ does not keep pace with $k^d$ as $k$ is increased then $c$ can be considered an outlier cluster.
    \end{itemize}
    \item Consider the connected components of the graph at a depth in the tree just before the graph shatters into many small components or isolated clusters.
    \begin{itemize}
        \item If a component contains too few points/clusters then those points/clusters can be considered outliers.
        \item If a small component is connected to a larger component with a small number of edges then the smaller component may contain outliers.
    \end{itemize}
\end{enumerate}

Due to the wide range of possible measurements for ``anomalousness'' from our methods, we normalize our measurements.

\subsection{Normalization Methods}

\begin{enumerate}
    \item Min-Max Scaling:
    \begin{gather}
        x^{\prime} = \frac{x - x_{min}}{x_{max} - x_{min}}
        \label{sec:methods:min-max-normalizationn}
    \end{gather}
    
    \item Mean-Scaling:
    \begin{gather}
        x^{\prime} = \frac{x - \overline{x}}{x_{max} - x_{min}}
        \label{sec:methods:min-max-normalizationn}
    \end{gather}
    
    \item z-Score Standardization:
    \begin{gather}
        x^{\prime} = \frac{x - \overline{x}}{\sigma}
        \label{sec:methods:min-max-normalizationn}
    \end{gather}

\end{enumerate}

\section{Results}
\label{sec:results}

We have implemented and tested our methods on 26 different real-world datasets.
All datasets studied were sourced from Outlier Detection Datasets (ODDS)~\cite{rayana2016odds}, which provides clear labels for normal and anomalous instances.
We examine the effectiveness of our methods by their ability to successfully identify these outliers.

For each dataset, we clustered using the Manhattan, Euclidean, and Cosine distance functions.
Additionally, for this work, we allowed CLAM to cluster down until the manifold had thoroughly ``shattered''.
We then iterated over all depths and computed the area under the ROC-Curve for each of our detection methods.

After computing anomalousness scores, we normalized these scores to a $[0, 1]$ range.
For each dataset and for each metric, we present a plot of the area under the ROC-curve as a function of depth.
We can see from these plots that the area under the ROC-curve is not overly sensitive to depth.

Table~\ref{results:tbl-properties} summarizes the datasets studied in this work.
As mentioned, each of these datasets were sourced from ODDS.
However, this is where their commonalities end.
Each dataset contains a different number of points, a different number of dimensions, and a different fraction of outliers.

\begin{small}
\begin{table}[!t]
\renewcommand{\arraystretch}{1.3}
\caption{Properties of the data sets analyzed}
\label{results:tbl-properties}
\centering
\begin{tabular}{|c|c|c|c|}
\hline
\bfseries Dataset & \bfseries \# Points & \bfseries \# Dim. & \bfseries \# Outliers \\ 
\hline
\bfseries lympho & 148 & 18 & 6 \\
\hline
\bfseries wbc & 278 & 30 & 21 \\
\hline
\bfseries glass & 214 & 9 & 9 \\
\hline
\bfseries vowels & 1,456 & 12 & 50 \\
\hline
\bfseries cardio & 1,831 & 21 & 176 \\
\hline
\bfseries thyroid & 3,772 & 6 & 93 \\
\hline
\bfseries musk & 3,062 & 166 & 166 \\
\hline
\bfseries satimage-2 & 5,803 & 36 & 71 \\
\hline
\bfseries pima & 768 & 8 & 268 \\
\hline
\bfseries satellite & 6,435 & 36 & 2,036 \\
\hline
\bfseries shuttle & 49,097 & 9 & 3,511 \\
\hline
\bfseries breastw & 683 & 9 & 239 \\
\hline
\bfseries arrhythmia & 452 & 274 & 66 \\
\hline
\bfseries ionosphere & 351 & 33 & 126 \\
\hline
\bfseries mnist & 7,063 & 100 & 700 \\
\hline
\bfseries optdigits & 5,216 & 64 & 150 \\
\hline
\bfseries http & 567,479 & 3 & 2,211 \\
\hline
\bfseries cover & 286,048 & 10 & 2,747 \\
\hline
\bfseries smtp & 95,156 & 3 & 30 \\
\hline
\bfseries mammography & 11,183 & 6 & 260 \\
\hline
\bfseries annthyroid & 7,200 & 6 & 534 \\
\hline
\bfseries pendigits & 6,870 & 16 & 156 \\
\hline
\bfseries wine & 129 & 13 & 10 \\
\hline
\bfseries vertebral & 240 & 6 & 30 \\
\hline
\end{tabular}
\label{description-table}
\end{table}
\end{small}


The performance of the algorithms presented here are detailed in Table~\ref{results:tbl-comparison}, along with state-of-the-art results from other works in the literature.
As can be observed, CHAODA performs comparably to all state-of-the-art algorithms on all datasets, with the sole exception of the annthyroid dataset.
In fact, in many cases, CHAODA outperforms other leading algorithms.

\begin{small}
\begin{table*}[!t]
\renewcommand{\arraystretch}{1.3}
\caption{Comparison of CHAODA with state-of-the-art algorithms}
\label{results:tbl-comparison}
\centering
\begin{tabular}{|c|c|c|c|c|c|c|c|c|c|c|c|c|c|c|c|}
\hline
\bfseries Dataset & \bfseries CC & \bfseries PC & \bfseries KN & \bfseries RW & \bfseries SC & \bfseries SVM & \bfseries LOF & \bfseries HiCS & \bfseries LODES & \bfseries iForest & \bfseries Mass & \bfseries MassE & \bfseries AOD & \bfseries HST & \bfseries iNNE \\ 
\hline
\bfseries lympho & 0.97 & 0.96 & 0.89 & 0.96 & 0.74 & 0.70 & - & - & - & - & - & - & - & - & - \\ 
\hline
\bfseries wbc & 0.95 & 0.96 & 0.67 & - & 0.57 & 0.78 & - & 0.59 & - & - & - & - & - & - & - \\ 
\hline
\bfseries glass & 0.87 & 0.80 & 0.71 & - & 0.74 & 0.58 & - & 0.80 & 0.87 & - & - & - & - & - & - \\ 
\hline
\bfseries vowels & 0.90 & 0.88 & 0.59 & - & 0.62 & 0.69 & - & - & 0.91 & - & - & - & - & - & - \\ 
\hline
\bfseries cardio & 0.86 & 0.87 & 0.74 & - & 0.77 & 0.78 & - & - & 0.72 & - & - & - & - & - & - \\ 
\hline
\bfseries thyroid & 0.94 & 0.94 & 0.88 & - & 0.87 & 0.78 & - & - & 0.68 & - & - & - & - & - & - \\ 
\hline
\bfseries musk & 1.00 & 1.00 & 1.00 & 1.00 & 1.00 & 0.74 & - & - & - & - & - & - & - & - & - \\ 
\hline
\bfseries satimage-2 & 0.99 & 0.99 & 0.94 & - & 0.94 & 0.76 & - & - & - & - & - & - & - & - & - \\ 
\hline
\bfseries pima & 0.69 & 0.68 & 0.66 & - & 0.63 & 0.61 & - & 0.72 & - & 0.67 & 0.69 & - & - & - & - \\ 
\hline
\bfseries satellite & 0.81 & 0.82 & 0.72 & - & 0.68 & 0.60 & - & - & - & 0.71 & 0.74 & 0.77 & - & - & - \\ 
\hline
\bfseries shuttle & 0.92 & 0.92 & 0.92 & - & 0.95 & 0.77 & 0.84 & - & - & 1.00 & 1.00 & 1.00 & 1.00 & 1.00 & 0.99 \\ 
\hline
\bfseries breastw & 0.97 & 0.96 & 0.55 & - & 0.51 & 0.85 & - & 0.94 & - & 0.99 & 0.99 & - & - & - & - \\ 
\hline
\bfseries arrhythmia & 0.75 & 0.77 & 0.69 & 0.79 & 0.55 & 0.68 & - & 0.62 & - & 0.80 & 0.84 & - & - & - & - \\ 
\hline
\bfseries ionosphere & 0.87 & 0.87 & 0.55 & - & 0.54 & 0.73 & - & 0.82 & - & 0.85 & 0.80 & - & - & - & - \\ 
\hline
\bfseries mnist & 0.85 & 0.84 & 0.53 & - & 0.53 & 0.73 & - & - & - & - & - & - & - & - & 0.87 \\ 
\hline
\bfseries optdigits & 0.78 & 0.78 & 0.59 & - & 0.71 & 0.50 & - & - & - & - & - & - & - & - & - \\
\hline
\bfseries http & 1.00 & 1.00 & 0.99 & - & 1.00 & 0.76 & - & - & - & 1.00 & 1.00 & 1.00 & 0.94 & 0.98 & 1.00 \\ 
\hline
\bfseries cover & 0.82 & 0.82 & 0.84 & - & 0.57 & 0.66 & - & - & - & 0.88 & 0.89 & 0.92 & - & 0.85 & 0.97 \\ 
\hline
\bfseries smtp & 0.91 & 0.91 & 0.87 & - & 0.85 & 0.69 & - & - & - & 0.88 & 0.90 & 0.91 & - & 0.74 & 0.95 \\ 
\hline
\bfseries mammography & 0.85 & 0.86 & 0.78 & - & 0.73 & 0.65 & - & - & - & 0.86 & 0.86 & 0.86 & 0.81 & - & - \\ 
\hline
\bfseries annthyroid & 0.69 & 0.74 & 0.64 & - & 0.66 & 0.55 & 0.87 & 0.95 & - & 0.82 & 0.73 & 0.75 & 0.97 & - & - \\ 
\hline
\bfseries pendigits & 0.95 & 0.95 & 0.78 & - & 0.89 & 0.75 & - & 0.95 & 0.94 & - & - & - & - & - & - \\ 
\hline
\bfseries wine & 1.00 & 1.00 & 0.99 & - & 0.99 & 0.72 & - & - & 0.97 & - & - & - & - & - & - \\ 
\hline
\bfseries vertebral & 0.58 & 0.52 & 0.70 & - & 0.71 & 0.54 & - & - & 0.58 & - & - & - & - & - & - \\ 
\hline
\end{tabular}
\end{table*}
\end{small}

In Figures~\ref{results:datasets_1},~\ref{results:datasets_2},~\ref{results:datasets_3}, and~\ref{results:datasets_4}, we express the ROC-AUC vs depth for our measures of anomalousness.
These figures demonstrate that in many circumstances our performance is not a matter of finding the singular depth at which we perform well, but rather that our methods are tolerant to minor variations in depth.
We can see that for the majority of datasets our algorithms performance increases at first, but as depth increases the manifolds begin to shatter, thus bringing performance back down.

\begin{figure*}[!t]
\centering
% Annthyroid
\includegraphics[width=2.2in]{kdd/static/auc_vs_depth/annthyroid-cosine.png}
\includegraphics[width=2.2in]{kdd/static/auc_vs_depth/annthyroid-euclidean.png}
\includegraphics[width=2.2in]{kdd/static/auc_vs_depth/annthyroid-manhattan.png}

% Arrhythmia
\includegraphics[width=2.2in]{kdd/static/auc_vs_depth/arrhythmia-cosine.png}
\includegraphics[width=2.2in]{kdd/static/auc_vs_depth/arrhythmia-euclidean.png}
\includegraphics[width=2.2in]{kdd/static/auc_vs_depth/arrhythmia-manhattan.png}

% BreastW
\includegraphics[width=2.2in]{kdd/static/auc_vs_depth/breastw-cosine.png}
\includegraphics[width=2.2in]{kdd/static/auc_vs_depth/breastw-euclidean.png}
\includegraphics[width=2.2in]{kdd/static/auc_vs_depth/breastw-manhattan.png}

% Cardio
\includegraphics[width=2.2in]{kdd/static/auc_vs_depth/cardio-cosine.png}
\includegraphics[width=2.2in]{kdd/static/auc_vs_depth/cardio-euclidean.png}
\includegraphics[width=2.2in]{kdd/static/auc_vs_depth/cardio-manhattan.png}

% Glass
\includegraphics[width=2.2in]{kdd/static/auc_vs_depth/glass-cosine.png}
\includegraphics[width=2.2in]{kdd/static/auc_vs_depth/glass-euclidean.png}
\includegraphics[width=2.2in]{kdd/static/auc_vs_depth/glass-manhattan.png}

% Ionosphere
\includegraphics[width=2.2in]{kdd/static/auc_vs_depth/ionosphere-cosine.png}
\includegraphics[width=2.2in]{kdd/static/auc_vs_depth/ionosphere-euclidean.png}
\includegraphics[width=2.2in]{kdd/static/auc_vs_depth/ionosphere-manhattan.png}

\caption{
Plots of ROC-AUC vs Depth for our measures of Anomolousness.
}

\label{results:datasets_1}
\end{figure*}

\begin{figure*}[!t]
\centering
% Lympho
\includegraphics[width=2.2in]{kdd/static/auc_vs_depth/lympho-cosine.png}
\includegraphics[width=2.2in]{kdd/static/auc_vs_depth/lympho-euclidean.png}
\includegraphics[width=2.2in]{kdd/static/auc_vs_depth/lympho-manhattan.png}

% Mnist
\includegraphics[width=2.2in]{kdd/static/auc_vs_depth/mnist-cosine.png}
\includegraphics[width=2.2in]{kdd/static/auc_vs_depth/mnist-euclidean.png}
\includegraphics[width=2.2in]{kdd/static/auc_vs_depth/mnist-manhattan.png}

% Musk
\includegraphics[width=2.2in]{kdd/static/auc_vs_depth/musk-cosine.png}
\includegraphics[width=2.2in]{kdd/static/auc_vs_depth/musk-euclidean.png}
\includegraphics[width=2.2in]{kdd/static/auc_vs_depth/musk-manhattan.png}

% Optdigits
\includegraphics[width=2.2in]{kdd/static/auc_vs_depth/optdigits-cosine.png}
\includegraphics[width=2.2in]{kdd/static/auc_vs_depth/optdigits-euclidean.png}
\includegraphics[width=2.2in]{kdd/static/auc_vs_depth/optdigits-manhattan.png}

% Pima
\includegraphics[width=2.2in]{kdd/static/auc_vs_depth/pima-cosine.png}
\includegraphics[width=2.2in]{kdd/static/auc_vs_depth/pima-euclidean.png}
\includegraphics[width=2.2in]{kdd/static/auc_vs_depth/pima-manhattan.png}

% Satellite
\includegraphics[width=2.2in]{kdd/static/auc_vs_depth/satellite-cosine.png}
\includegraphics[width=2.2in]{kdd/static/auc_vs_depth/satellite-euclidean.png}
\includegraphics[width=2.2in]{kdd/static/auc_vs_depth/satellite-manhattan.png}

\caption{
Plots of ROC-AUC vs Depth for our measures of Anomolousness.
}

\label{results:datasets_2}
\end{figure*}

\begin{figure*}[!t]
\centering
% Satimage-2
\includegraphics[width=2.2in]{kdd/static/auc_vs_depth/satimage-2-cosine.png}
\includegraphics[width=2.2in]{kdd/static/auc_vs_depth/satimage-2-euclidean.png}
\includegraphics[width=2.2in]{kdd/static/auc_vs_depth/satimage-2-manhattan.png}

% Thyroid
\includegraphics[width=2.2in]{kdd/static/auc_vs_depth/thyroid-cosine.png}
\includegraphics[width=2.2in]{kdd/static/auc_vs_depth/thyroid-euclidean.png}
\includegraphics[width=2.2in]{kdd/static/auc_vs_depth/thyroid-manhattan.png}

% Vertebral
\includegraphics[width=2.2in]{kdd/static/auc_vs_depth/vertebral-cosine.png}
\includegraphics[width=2.2in]{kdd/static/auc_vs_depth/vertebral-euclidean.png}
\includegraphics[width=2.2in]{kdd/static/auc_vs_depth/vertebral-manhattan.png}

% Vowels
\includegraphics[width=2.2in]{kdd/static/auc_vs_depth/vowels-cosine.png}
\includegraphics[width=2.2in]{kdd/static/auc_vs_depth/vowels-euclidean.png}
\includegraphics[width=2.2in]{kdd/static/auc_vs_depth/vowels-manhattan.png}

% WBC
\includegraphics[width=2.2in]{kdd/static/auc_vs_depth/wbc-cosine.png}
\includegraphics[width=2.2in]{kdd/static/auc_vs_depth/wbc-euclidean.png}
\includegraphics[width=2.2in]{kdd/static/auc_vs_depth/wbc-manhattan.png}

% Wine
\includegraphics[width=2.2in]{kdd/static/auc_vs_depth/wine-cosine.png}
\includegraphics[width=2.2in]{kdd/static/auc_vs_depth/wine-euclidean.png}
\includegraphics[width=2.2in]{kdd/static/auc_vs_depth/wine-manhattan.png}

\caption{
Plots of ROC-AUC vs Depth for our measures of Anomolousness.
}

\label{results:datasets_3}
\end{figure*}

\begin{figure*}[!t]
\centering

% HTTP
\includegraphics[width=2.2in]{kdd/static/auc_vs_depth/http-cosine.png}
\includegraphics[width=2.2in]{kdd/static/auc_vs_depth/http-euclidean.png}
\includegraphics[width=2.2in]{kdd/static/auc_vs_depth/http-manhattan.png}

% Shuttle
\includegraphics[width=2.2in]{kdd/static/auc_vs_depth/shuttle-cosine.png}
\includegraphics[width=2.2in]{kdd/static/auc_vs_depth/shuttle-euclidean.png}
\includegraphics[width=2.2in]{kdd/static/auc_vs_depth/shuttle-manhattan.png}

% SMTP
\includegraphics[width=2.2in]{kdd/static/auc_vs_depth/smtp-cosine.png}
\includegraphics[width=2.2in]{kdd/static/auc_vs_depth/smtp-euclidean.png}
\includegraphics[width=2.2in]{kdd/static/auc_vs_depth/smtp-manhattan.png}

% Pendigits
\includegraphics[width=2.2in]{kdd/static/auc_vs_depth/pendigits-cosine.png}
\includegraphics[width=2.2in]{kdd/static/auc_vs_depth/pendigits-euclidean.png}
\includegraphics[width=2.2in]{kdd/static/auc_vs_depth/pendigits-manhattan.png}

% Mammography
\includegraphics[width=2.2in]{kdd/static/auc_vs_depth/mammography-cosine.png}
\includegraphics[width=2.2in]{kdd/static/auc_vs_depth/mammography-euclidean.png}
\includegraphics[width=2.2in]{kdd/static/auc_vs_depth/mammography-manhattan.png}

\caption{
Plots of ROC-AUC vs Depth for our measures of Anomolousness.
}

\label{results:datasets_4}
\end{figure*}






\begin{figure*}[!t]
\centering
% Annthyroid
\includegraphics[width=2.2in]{kdd/static/lfd_vs_depth/annthyroid-cosine.png}
\includegraphics[width=2.2in]{kdd/static/lfd_vs_depth/annthyroid-euclidean.png}
\includegraphics[width=2.2in]{kdd/static/lfd_vs_depth/annthyroid-manhattan.png}

% Arrhythmia
\includegraphics[width=2.2in]{kdd/static/lfd_vs_depth/arrhythmia-cosine.png}
\includegraphics[width=2.2in]{kdd/static/lfd_vs_depth/arrhythmia-euclidean.png}
\includegraphics[width=2.2in]{kdd/static/lfd_vs_depth/arrhythmia-manhattan.png}

% BreastW
\includegraphics[width=2.2in]{kdd/static/lfd_vs_depth/breastw-cosine.png}
\includegraphics[width=2.2in]{kdd/static/lfd_vs_depth/breastw-euclidean.png}
\includegraphics[width=2.2in]{kdd/static/lfd_vs_depth/breastw-manhattan.png}

% Cardio
\includegraphics[width=2.2in]{kdd/static/lfd_vs_depth/cardio-cosine.png}
\includegraphics[width=2.2in]{kdd/static/lfd_vs_depth/cardio-euclidean.png}
\includegraphics[width=2.2in]{kdd/static/lfd_vs_depth/cardio-manhattan.png}

% Glass
\includegraphics[width=2.2in]{kdd/static/lfd_vs_depth/glass-cosine.png}
\includegraphics[width=2.2in]{kdd/static/lfd_vs_depth/glass-euclidean.png}
\includegraphics[width=2.2in]{kdd/static/lfd_vs_depth/glass-manhattan.png}

% Ionosphere
\includegraphics[width=2.2in]{kdd/static/auc_vs_depth/ionosphere-cosine.png}
\includegraphics[width=2.2in]{kdd/static/auc_vs_depth/ionosphere-euclidean.png}
\includegraphics[width=2.2in]{kdd/static/auc_vs_depth/ionosphere-manhattan.png}

\caption{
Plots of Local Fractal Dimension vs Depth for our datasets.
}

\label{results:lfd_1}
\end{figure*}

\begin{figure*}[!t]
\centering
% Lympho
\includegraphics[width=2.2in]{kdd/static/lfd_vs_depth/lympho-cosine.png}
\includegraphics[width=2.2in]{kdd/static/lfd_vs_depth/lympho-euclidean.png}
\includegraphics[width=2.2in]{kdd/static/lfd_vs_depth/lympho-manhattan.png}

% Mnist
\includegraphics[width=2.2in]{kdd/static/lfd_vs_depth/mnist-cosine.png}
\includegraphics[width=2.2in]{kdd/static/lfd_vs_depth/mnist-euclidean.png}
\includegraphics[width=2.2in]{kdd/static/lfd_vs_depth/mnist-manhattan.png}

% Musk
\includegraphics[width=2.2in]{kdd/static/lfd_vs_depth/musk-cosine.png}
\includegraphics[width=2.2in]{kdd/static/lfd_vs_depth/musk-euclidean.png}
\includegraphics[width=2.2in]{kdd/static/lfd_vs_depth/musk-manhattan.png}

% Optdigits
\includegraphics[width=2.2in]{kdd/static/lfd_vs_depth/optdigits-cosine.png}
\includegraphics[width=2.2in]{kdd/static/lfd_vs_depth/optdigits-euclidean.png}
\includegraphics[width=2.2in]{kdd/static/lfd_vs_depth/optdigits-manhattan.png}

% Pima
\includegraphics[width=2.2in]{kdd/static/lfd_vs_depth/pima-cosine.png}
\includegraphics[width=2.2in]{kdd/static/lfd_vs_depth/pima-euclidean.png}
\includegraphics[width=2.2in]{kdd/static/lfd_vs_depth/pima-manhattan.png}

% Satellite
\includegraphics[width=2.2in]{kdd/static/lfd_vs_depth/satellite-cosine.png}
\includegraphics[width=2.2in]{kdd/static/lfd_vs_depth/satellite-euclidean.png}
\includegraphics[width=2.2in]{kdd/static/lfd_vs_depth/satellite-manhattan.png}

\caption{
Plots of Local Fractal Dimension vs Depth for our datasets.
}

\label{results:lfd_2}
\end{figure*}

\begin{figure*}[!t]
\centering
% Satimage-2
\includegraphics[width=2.2in]{kdd/static/lfd_vs_depth/satimage-2-cosine.png}
\includegraphics[width=2.2in]{kdd/static/lfd_vs_depth/satimage-2-euclidean.png}
\includegraphics[width=2.2in]{kdd/static/lfd_vs_depth/satimage-2-manhattan.png}

% Thyroid
\includegraphics[width=2.2in]{kdd/static/lfd_vs_depth/thyroid-cosine.png}
\includegraphics[width=2.2in]{kdd/static/lfd_vs_depth/thyroid-euclidean.png}
\includegraphics[width=2.2in]{kdd/static/lfd_vs_depth/thyroid-manhattan.png}

% Vertebral
\includegraphics[width=2.2in]{kdd/static/lfd_vs_depth/vertebral-cosine.png}
\includegraphics[width=2.2in]{kdd/static/lfd_vs_depth/vertebral-euclidean.png}
\includegraphics[width=2.2in]{kdd/static/lfd_vs_depth/vertebral-manhattan.png}

% Vowels
\includegraphics[width=2.2in]{kdd/static/lfd_vs_depth/vowels-cosine.png}
\includegraphics[width=2.2in]{kdd/static/lfd_vs_depth/vowels-euclidean.png}
\includegraphics[width=2.2in]{kdd/static/lfd_vs_depth/vowels-manhattan.png}

% WBC
\includegraphics[width=2.2in]{kdd/static/lfd_vs_depth/wbc-cosine.png}
\includegraphics[width=2.2in]{kdd/static/lfd_vs_depth/wbc-euclidean.png}
\includegraphics[width=2.2in]{kdd/static/lfd_vs_depth/wbc-manhattan.png}

% Wine
\includegraphics[width=2.2in]{kdd/static/lfd_vs_depth/wine-cosine.png}
\includegraphics[width=2.2in]{kdd/static/lfd_vs_depth/wine-euclidean.png}
\includegraphics[width=2.2in]{kdd/static/lfd_vs_depth/wine-manhattan.png}

\caption{
Plots of Local Fractal Dimension vs Depth for our datasets.
}

\label{results:lfd_3}
\end{figure*}

\begin{figure*}[!t]
\centering

% HTTP
\includegraphics[width=2.2in]{kdd/static/lfd_vs_depth/http-cosine.png}
\includegraphics[width=2.2in]{kdd/static/lfd_vs_depth/http-euclidean.png}
\includegraphics[width=2.2in]{kdd/static/lfd_vs_depth/http-manhattan.png}

% Shuttle
\includegraphics[width=2.2in]{kdd/static/lfd_vs_depth/shuttle-cosine.png}
\includegraphics[width=2.2in]{kdd/static/lfd_vs_depth/shuttle-euclidean.png}
\includegraphics[width=2.2in]{kdd/static/lfd_vs_depth/shuttle-manhattan.png}

% SMTP
\includegraphics[width=2.2in]{kdd/static/lfd_vs_depth/smtp-cosine.png}
\includegraphics[width=2.2in]{kdd/static/lfd_vs_depth/smtp-euclidean.png}
\includegraphics[width=2.2in]{kdd/static/lfd_vs_depth/smtp-manhattan.png}

% Pendigits
\includegraphics[width=2.2in]{kdd/static/lfd_vs_depth/pendigits-cosine.png}
\includegraphics[width=2.2in]{kdd/static/lfd_vs_depth/pendigits-euclidean.png}
\includegraphics[width=2.2in]{kdd/static/lfd_vs_depth/pendigits-manhattan.png}

% Mammography
\includegraphics[width=2.2in]{kdd/static/lfd_vs_depth/mammography-cosine.png}
\includegraphics[width=2.2in]{kdd/static/lfd_vs_depth/mammography-euclidean.png}
\includegraphics[width=2.2in]{kdd/static/lfd_vs_depth/mammography-manhattan.png}

\caption{
Plots of Local Fractal Dimension vs Depth for our datasets.
}

\label{results:lfd_4}
\end{figure*}



\clearpage
\section{Conclusions}
\label{sec:conclusions}

We have presented CHAODA (Clustered Hierarchical Anomaly and Outlier Detection Algorithms), a collection of five algorithms that share the property of exploiting properties of a hierarchical cluster tree which learns a manifold in potentially high-dimensional space.
All five algorithms are trivial to implement on top of a manifold-learning framework we call CLAM (Clustered Learning of Approximate Manifolds); CHAODA builds on this framework just like CHESS~\cite{ishaq2019entropy}, which learns a manifold in the same way, but for the purpose of accelerating approximate search.
In CHESS, the geometric and topological properties of low fractal dimension and low metric entropy are advantages; indeed, CHESS does not perform particularly well if those properties are not present.
CHAODA, on the other hand, while competitive with other state-of-the-art anomaly-detection approaches on ``easy'' data sets (we define as \textit{easy} any data set where a one-class SVM performs well), outperforms other current methods when the data exhibit precisely those properties that CHESS depends on for acceleration.

% note: make sure the above statement is actually true! Need to look at t-SNE and UMAP 
% for the breast cancer data set, as well as fractal dimension

Some discussion about where CHAODA doesn't perform as well...

One current limitation in CHAODA is that the depth of the cluster tree at which anomaly detection performs best is not the same for every data set, and thus our results could be seen as ``cherry-picking'' from a scattershot approach.
Future work should certainly explore optimal stopping criteria so that this process can be further automated.
However, one can clearly observe that the choice of depth is robust to minor deviations, and one can treat depth as a hyperparameter to the methods described.

% TODO: Need to look at local fractal dimension, or volume ratios, vs. optimal depth. Ideally we can say something like:

The strong correlation between local fractal dimension and optimal tree depth suggests a guideline for determining an optimal tree depth directly from the data.

The choice of distance function can also have a significant impact on anomaly-detection performance.
In this case, domain knowledge is likely the best way to determine the distance function of choice.
In future work, we seek to explore a more diverse collection of domain-appropriate distance functions, such as Wasserstein distance on images, or Levenshtein edit distance on strings. 

% TODO: Have we proved this?
Say something about applying CHAODA for inputs to an ANN, in particular detecting just-off-manifold malicious inputs, like the school bus / ostrich example.


\printbibliography
\end{document}
